% !TEX program = xelatex
\documentclass[aspectratio=169,utf8]{ctexbeamer}
% use beamer if english only
% aspectratio: 1610, 169, 149, 54, 43 and 32
% By default, it is to 128mm by 96mm, 4:3

\usepackage[english]{babel}
\usepackage{graphicx,hyperref,googleblue}
\usepackage{xcolor}
\usepackage{braket}
\usepackage{listings}

\lstset{
  upquote,
  keepspaces=true,
  columns=spaceflexible,
  numbers=left,
  basicstyle=\ttfamily\scriptsize,
  numberstyle=\color{gray}\ttfamily\tiny,
  keywordstyle=\color{blue}\ttfamily,
  stringstyle=\color{red}\ttfamily,
  commentstyle=\color{teal}\ttfamily,
  emphstyle=\color{blue}\bfseries,
  frame=none
}

\usepackage{multicol}
\setlength\columnsep{10pt} % This is the default columnsep

\usepackage{setspace}

%\usepackage{ebgaramond}
\usefonttheme{serif}
\usepackage{fontspec}
%\setmainfont{Helvetica Neue}
\setmainfont{Arial}
\setCJKmainfont{Microsoft YaHei}
%\setCJKmainfont[BoldFont=STHeiti,ItalicFont=STKaiti]{STHeiti}
%\setCJKsansfont[BoldFont=STHeiti]{STXihei}
%\setCJKmonofont{STKaiti}

\setbeamercovered{transparent}


\title[Beamer Tutorial]{Beamer in a Nutshell}
\subtitle{presentation with \LaTeX{} made easy}
\author[WisdomFusion]{\scriptsize \textcolor{gray}{WisdomFusion\\ <WisdomFusion@gmail.com>}}
\date{May, $2018$}


\begin{document}

\addtocounter{framenumber}{-1}


\begin{frame}
  \titlepage
\end{frame}

\begin{frame}
  \frametitle{Outline}

  \centering

  \begin{minipage}{.75\textwidth}
    % \doublespacing
    % \onehalfspacing
    \setstretch{1.5}

    \begin{columns}[t]
      \begin{column}{.5\textwidth}
        \tableofcontents[sections={1-6}]
      \end{column}
      \begin{column}{.5\textwidth}
        \tableofcontents[sections={7-12}]
      \end{column}      
    \end{columns}

    \setstretch{1.5}
    % \singlespacing
  \end{minipage}

\end{frame}


\section{Introduction}
\begin{frame}
  \frametitle{What's Beamer?}

  \begin{itemize}
    \setlength{\itemsep}{8pt}
    \item Beamer is a fexible \LaTeX{} class for making slides and presentations.
    \item It supports functionality for making PDF slides complete with colors, overlays, environments, themes, transitions, etc.
    \item Adds a couple new features to the commands you've been working with.
  \end{itemize}
  
\end{frame}


\begin{frame}
  \frametitle{Advantages of Beamer}

  \begin{itemize}
    \setlength{\itemsep}{8pt}
    \item The standard commands of \LaTeX{} also work in Beamer. If you can write basic \LaTeX{}, you can easily make a Beamer presentation.
    \item You can easily create overlays, themes allow you to change the appearance of your presentation to suit your purposes.
    \item The layout, colors, and fonts used in a presentation can easily be changed globally, but you also have control over the most minute detail.
  \end{itemize}
  
\end{frame}

\begin{frame}
  \frametitle{Advantages of Beamer}

  \begin{itemize}
    \setlength{\itemsep}{8pt}
    \item Each theme is designed to be highly usable and readable. This makes the presentation more professional looking and easier for the audience to follow.
    \item The final output is typically a \alert{.pdf} file. Viewer applications for this format exist for virtually every platform.
    \item \alert{Your presentation will look exactly the same no matter which computer or viewer program is being used.}
  \end{itemize}
  
\end{frame}

\section{Template}
\begin{frame}[fragile]
  \frametitle{Template}

  \begin{columns}[t]
    \begin{column}{.5\textwidth}

\begin{lstlisting}[language=TeX]
% !TEX program = xelatex
\documentclass[aspectratio=169,utf8]{ctexbeamer}

\usepackage{graphicx,hyperref}
\usepackage{xcolor}
\usefonttheme{serif}
\usepackage{fontspec}
\setmainfont{Helvetica Neue}
\setCJKmainfont{PingFang SC}

\title[short title]{long title}
\subtitle[short subtitle]{long subtitle}
\author[short name]{long name}

\end{lstlisting}

        $\longrightarrow$
    \end{column}

    \begin{column}{.35\textwidth}

\begin{lstlisting}[language=TeX, firstnumber=14]
% \begin{document}
% \begin{frame}
%   \titlepage
% \end{frame}
% \begin{frame}
%   \frametitle{Outline}
%   \tableofcontents
% \end{frame}
% \section{Some Section}
% \begin{frame}
%   \frametitle{Section Title}
% 
%   Section content
% \end{frame}
% \end{document}
\end{lstlisting}
      
    \end{column}
    
  \end{columns}

\end{frame}

\begin{frame}
  \frametitle{Insert Title Information}

  \begin{block}{Commands To Change}
    \tt
    \begin{itemize}
      \item
        \textbackslash{}title[\alert{short title}]\{\alert{long title}\}
      \item
        \textbackslash{}subtitle[\alert{short subtitle}]\{\alert{long subtitle}\}
      \item
        \textbackslash{}author[\alert{short name}]\{\alert{long name}\}
      \item
        \textbackslash{}date[\alert{short date}]\{\alert{long date}\}
      \item
        \textbackslash{}institution[\alert{short name}]\{\alert{long name}\}
    \end{itemize}
  \end{block}
    
\end{frame}

\section{Frames}
\begin{frame}
  \frametitle{Frames}

  \begin{itemize}
    \setlength{\itemsep}{8pt}
    \item Each Beamer project is made up of a series of frames.
    \item Each frame produces one or more slides, depending on the slide’s overlays, which will be discussed later.
  \end{itemize}

%A Basic Frame
%\begin{frame}[<alignment>]
%\frametitle{Frame Title Goes Here}
%Frame body text and/or LATEX code
%\end{frame}
      
\end{frame}


\begin{frame}
  \frametitle{Frames}

  Frames are very simple to make. Simply write your own text or \LaTeX{} code between the begin/end frame commands.

  The alignment option is centered {\tt [c]} by default. The values {\tt [t]} (top align) and {\tt [b]} (bottom align) are also accepted.

%A Basic Frame
%\begin{frame}[t]
%\frametitle{Algorithmic Combinatorics on Words}
%\textit{Words}, or strings of symbols over..
%\end{frame}

\end{frame}


\begin{frame}[fragile]
  \frametitle{Frames}

  \begin{itemize}
    \setlength{\itemsep}{8pt}
    \item The {\tt [plain]} option for the frame environment causes the headlines, footlines, and sidebars to be suppressed. This can be useful for showing large pictures.
    \item If you already have a \LaTeX{} document, you can simply wrap {\tt \verb|\begin{frame}|} and {\tt \verb|\end{frame}|} commands around the information you want to present.
  \end{itemize}
\end{frame}

\section{Sections}

\begin{frame}[fragile]
  \frametitle{Sections and Subsections}

  \begin{itemize}
    \setlength{\itemsep}{8pt}
    \item Presentations are divided into sections, subsections, and sub-subsections.
    \item Each call to the {\tt \verb|\section{section name}|}, {\tt \verb|\subsection{subsection name}|}, or {\tt \verb|\subsubsection{sub-subsection name}|} command:
      \begin{itemize}
        \item Inserts a new entry into the table of contents at the appropriate tree-level.
        \item Inserts a new entry into the navigation bars.
        \item Does not create a frame heading.
      \end{itemize}
    \item Another version of the command, {\tt \verb|\subsection*{section name}|}, only adds an entry in the navigation bars, \textit{not} the table of contents.
  \end{itemize}

\end{frame}

\begin{frame}
  \frametitle{Sections and Subsections}

  Section specifications are declared between the frames, so they have no direct effect on what is shown inside each frame.

%Example
%...
%\end{frame}
%\section{Fine and Wilf’s Theorem}
%\subsection{The Case of Two or Three Holes}
%\subsubsection{Definition 3.7}
%\begin{frame}
%...

\end{frame}


\section{Text}
\begin{frame}
  \frametitle{Common Text Commands and Environments}

  You can use the same text commands and environments in Beamer that you do in \LaTeX{} to change the way your text is displayed.

  \begin{block}{Common Text Commands}
    TODO
  \end{block}
\end{frame}

\begin{frame}
  \frametitle{Verbatim Text}
  
\end{frame}

\begin{frame}
  \frametitle{Semiverbatim Text}
  
\end{frame}


\section{Styles}

\begin{frame}
  \frametitle{Font Size}
  
\end{frame}

\begin{frame}
  \frametitle{Font Families}
  
\end{frame}

\begin{frame}
  \frametitle{Alightment}
  align
\end{frame}

\begin{frame}
  \frametitle{Spacing}
  spacing
\end{frame}


\section{Lists}
\begin{frame}
  \frametitle{Lists}
  lists
\end{frame}

\begin{frame}
  \frametitle{Lists - Itemize}
\end{frame}

\begin{frame}
  \frametitle{Lists - Enumerate}

  
\end{frame}

\begin{frame}
  \frametitle{Lists - Description}
\end{frame}


\section{Overlays}
\begin{frame}
  \frametitle{Overlays}
\end{frame}

\begin{frame}
  \frametitle{Overlays - Pause}
\end{frame}

\begin{frame}
  \frametitle{Overlays - Specifications}
\end{frame}

\section{Tables}
\begin{frame}
  \frametitle{Tables}
  tables
\end{frame}

\section{Structures}

\begin{frame}
  \frametitle{Blocks}
\end{frame}

\begin{frame}
  \frametitle{Columns}
\end{frame}

\begin{frame}
  \frametitle{Text Boxes}
\end{frame}

\section{Graphics}
\begin{frame}
  \frametitle{Graphics}
  graph
\end{frame}


\section{Transition}
\begin{frame}
  \frametitle{Transition}
  transition
\end{frame}

\addtocounter{framenumber}{-1}


\setbeamercolor{background canvas}{bg=matblue}
\setbeamercolor{normal text}{fg=white}
\setbeamertemplate{navigation symbols}{}
\begin{frame}[plain, b]
  \centering
  \Large \textcolor{white}{Thanks.}
  \normalsize

  \vspace*{\fill}

  \begin{beamercolorbox}[wd=\paperwidth]{section in head/foot}
    \centering
    \vskip3pt
    {\small Beamer -- powerful, flexible and nice-looking presentations}
    \vskip8pt
  \end{beamercolorbox}
  
\end{frame}


\end{document}


%%% Local Variables:
%%% mode: latex
%%% TeX-master: t
%%% End:

% !TEX program = xelatex
\documentclass[aspectratio=169,utf8]{ctexbeamer}
% use beamer if english only
% aspectratio: 1610, 169, 149, 54, 43 and 32
% By default, it is to 128mm by 96mm, 4:3

\usepackage{fontspec}
\usepackage[english]{babel}
\usepackage{graphicx,hyperref,googleblue}
\usepackage{xcolor}
\usepackage{braket}
\usepackage{listings}

\graphicspath{ {./figs/} }
\setbeamercovered{transparent}


\title[Beamer Tutorial]{
  A Beamer Tutorial in Beamer
}
\subtitle{presentation made easy}
\author[WisdomFusion]{
  \scriptsize \textcolor{gray}{WisdomFusion\\ <WisdomFusion@gmail.com>}
}
\date{May, $2018$}


\begin{document}

\addtocounter{framenumber}{-1}


\begin{frame}
  \titlepage
\end{frame}

\begin{frame}
  \frametitle{Outline}
  \tableofcontents
\end{frame}


\section*{Introduction}
\begin{frame}[t]
  \frametitle{About Beamer}

  \begin{block}{What is Beamer?}
    Beamer,
  \end{block}
  
\end{frame}


\begin{frame}
  \frametitle{Advantages of Beamer}

  \begin{itemize}
    \item The standard commands of \LaTeX{} also work in Beamer. If you can write basic \LaTeX{}, you can easily make a Beamer presentation.
    \item A table of contents will automatically be created, complete with clickable links to each section and subsection you create in your presentation.
    \item The layout, colors, and fonts used in a presentation can easily be changed globally, but you also have control over the most minute detail.
  \end{itemize}
  
\end{frame}

\section*{Template}
\begin{frame}
  \frametitle{Template}

  template

\end{frame}

\section*{Frame}
\begin{frame}
  \frametitle{Frame}

  frame
\end{frame}

\section*{Sections}
\begin{frame}
  \frametitle{Sections and Subsections}

  sections and subsections
\end{frame}

\section*{Text}
\begin{frame}
  \frametitle{Text}
  Text
\end{frame}

\section*{Styles}
\begin{frame}
  \frametitle{Alightment and Spacing}
  align
\end{frame}

\section*{Lists}
\begin{frame}
  \frametitle{Lists}
  lists
\end{frame}

\section*{Tables}
\begin{frame}
  \frametitle{Tables}
  tables
\end{frame}

\section*{Graphics}
\begin{frame}
  \frametitle{Graphics}
  graph
\end{frame}

\section*{Themes}
\begin{frame}
  \frametitle{Themes}
  themes
\end{frame}

\section*{Transition}
\begin{frame}
  \frametitle{Transition}
  transition
\end{frame}

\addtocounter{framenumber}{-1}


\setbeamercolor{background canvas}{bg=matblue}
\setbeamercolor{normal text}{fg=white}
\setbeamertemplate{navigation symbols}{}
\begin{frame}[plain, b]
  \centering
  \Large \textcolor{white}{Thanks.}
  \normalsize

  \vspace*{\fill}

  \begin{beamercolorbox}[wd=\paperwidth]{section in head/foot}
    \centering
    \vskip3pt
    {\small Beamer -- powerful, flexible and nice-looking presentations}
    \vskip8pt
  \end{beamercolorbox}
  
\end{frame}


\end{document}


%%% Local Variables:
%%% mode: latex
%%% TeX-master: t
%%% End:
